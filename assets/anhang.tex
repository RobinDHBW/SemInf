\setcounter{chapter}{23}
\chapter{Anhang}
\pagenumbering{Roman}

\section{Basis Kapillarrheometer}
Quelle: GÖTTFERT Werkstoff-Prüfmaschinen GmbH
\begin{figure}[ht]
    \begin{center}
        \includegraphics[width=0.75\textwidth]{assets/img/DE_IMG_Prinzip_Kapillarrheometer.png}
        \caption{Basis Kapillarrheometer}
        \label{base_capillary_rheometer}
    \end{center}
\end{figure}
\clearpage

\section{RHEOGRAPH 50}
Quelle: GÖTTFERT Werkstoff-Prüfmaschinen GmbH
\begin{figure}[ht]
    \begin{center}
        \includegraphics[width=0.75\textwidth]{assets/img/IMG_RG50_perspektive.png}
        \caption{RHEOGRAPH 50}
        \label{rg50}
    \end{center}
\end{figure}
\clearpage

% \section{Detaillierte Zeitplanung nach Phasen}
\begin{table}[ht]
    \begin{center}
        \begin{tabular}{| l | r  r |}
            \hline
            \textbf{Projektphase}                                & \multicolumn{2}{ c |}{\textbf{Zeit in h}}                \\
            \hline

            \textbf{Planungsphase}                               &                                           & \textbf{80} \\
            \hline
            \tabindent{IST-Analyse}                              & 20                                        &              \\
            \hline
            \tabindent{SOLL-Konzept}                             & 20                                        &              \\
            \hline
            \tabindent{Ablaufplanung}                            & 20                                        &              \\
            \hline
            \tabindent{Zeit- und Ressourcenplanung}              & 20                                        &              \\
            \hline

            \textbf{Durchführung}                                &                                           & \textbf{280} \\
            \hline
            \tabindent{Automation Mess- und Kalibrierprozess}    &                                           & 160          \\
            \hline
            \doubletabindent{Ablauf Temperaturmessung}           & 80                                       &              \\
            \hline
            \doubletabindent{Datenspeicherung}                   & 80                                        &              \\
            \hline
            \tabindent{Datentransfer und Auswertung}             &                                           & 120          \\
            \hline
            \doubletabindent{Datentransfer}                      & 40                                        &              \\
            \hline
            \doubletabindent{Integration PowerTool}              & 40                                        &              \\
            \hline
            \doubletabindent{Ermittlung Korrekturwert}           & 20                                        &              \\
            \hline
            \doubletabindent{Erstellung Prozessprotokoll}        & 20                                        &              \\
            \hline

            \textbf{Testphase/Abnahme}                           &                                           & \textbf{40}  \\
            \hline
            \tabindent{Testen einzelner Komponenten}             & 16                                        &              \\
            \hline
            \tabindent{Testen im Livebetrieb}                    & 16                                         &              \\
            \hline
            \tabindent{Abnahme durch Projektteam}                & 8                                        &              \\
            \hline

            \textbf{Auswertung/Fazit/Dokumentation}              &                                           & \textbf{80} \\
            \hline
            \tabindent{Auswertung: Gewinn durch Automatisierung} & 8                                        &              \\
            \hline
            \tabindent{Projekt-Dokumentation}                    & 72                                        &              \\

            \hline \hline
            \textbf{Gesamt}                                      &                                           & \textbf{480} \\
            \hline
        \end{tabular}
    \end{center}
    \caption{Detaillierte Zeitplanung nach Phasen}
    \label{zeitplanung-detail}
\end{table}
\clearpage


% \section{ISO 11443 2021}
% % \begin{figure}[ht]
%     % \begin{center}
%         \includepdf[pages=-]{assets/pdf/ISO 11443_2021.pdf}
%         % \includegraphics[width=0.9\textwidth, page=]{assets/pdf/ISO 11443_2021.pdf}
%         % \caption{ISO 11443 2021}
%         % \label{iso_11443_2021}
%     % \end{center}
% % \end{figure}
% \clearpage

\section{Temperaturkalibrierung und -justierung Hauptprozess}
Quelle: GÖTTFERT Werkstoff-Prüfmaschinen GmbH
\begin{figure}[ht]
    \begin{center}
        \includegraphics[width=1.0\textwidth]{assets/img/Temperaturkalibrierung RG.png}
        \caption{Temperaturkalibrierung und -justierung BPMN - Hauptprozess}
        \label{Temperaturkalibrierung_RG_Haupt}
    \end{center}
\end{figure}
\clearpage

\section{Unterprozess Korrekturwertbestimmung}
Quelle: GÖTTFERT Werkstoff-Prüfmaschinen GmbH
\begin{figure}[ht]
    \begin{center}
        \includegraphics[width=1.0\textwidth]{assets/img/Korrekturwerte bestimmen.png}
        \caption{Bestimmung Korrekturwerte BPMN}
        \label{Temperaturkalibrierung_RG_Sub_Korr}
    \end{center}
\end{figure}
\clearpage

\section{Unterprozess Temperaturkalibrierung}
Quelle: GÖTTFERT Werkstoff-Prüfmaschinen GmbH
\begin{figure}[ht]
    \begin{center}
        \includegraphics[width=1.0\textwidth]{assets/img/Temperaturwerte ziehen.png}
        \caption{Temperaturkalibrierung BPMN}
        \label{Temperaturkalibrierung_RG_Sub_Temp}
    \end{center}
\end{figure}
\clearpage

\section{Zeichnung Heißkanal RHEOGRAPH}\label{drawing_chamber_channel}
Quelle: GÖTTFERT Werkstoff-Prüfmaschinen GmbH
%  \begin{figure}[ht]
%     \begin{center}
\includepdf[pages=-, landscape=true]{assets/pdf/012.06.0.04.628.0_DOKU.pdf}

%  \includegraphics[width=1.4\textwidth, angle=90]{assets/pdf/012.06.0.04.628.0_DOKU.pdf}
%  \caption{Zeichnung Heißkanal RHEOGRAPH}

%     \end{center}
%  \end{figure}
\clearpage

\section{Dostmann P755 externes Temperaturmessgerät}
Quelle: \href{https://dostmann-electronic.de/gallery/1208gross_p455_p755_log_multifunktionsgeryt_fyr_temperatur_feuchte_strymung_druck.gif}{https://dostmann-electronic.de/}
\begin{figure}[ht]
    \begin{center}
        \includegraphics[width=0.6\textwidth]{assets/img/1208gross_p455_p755_log_multifunktionsgeryt_fyr_temperatur_feuchte_strymung_druck.png}
        \caption{Dostmann P755}
        \label{dostmann_p755}
    \end{center}
\end{figure}
\clearpage

\section{Ahlborn Almemo 202 externes Temperaturmessgerät}
Quelle: \href{https://www.ahlborn.com/pictures/202-03-557.png}{https://www.ahlborn.com/}
\begin{figure}[ht]
    \begin{center}
        \includegraphics[width=0.6\textwidth]{assets/img/202-03-557.png}
        \caption{Ahlborn Almemo 202}
        \label{almemo_202}
    \end{center}
\end{figure}
\clearpage

\section{GÖTTFERT Werkskalibrierschein für RHEOGRAPH}\label{wks_rheograph}
Quelle: GÖTTFERT Werkstoff-Prüfmaschinen GmbH
\includepdf[pages=-, landscape=true]{assets/pdf/WKS_002_BeHi_1000xxxx.pdf}
\clearpage

\section{BPMN automatisierter Ablauf Temperaturmessung}
Quelle: GÖTTFERT Werkstoff-Prüfmaschinen GmbH
\begin{figure}[ht]
    \begin{center}
        \includegraphics[width=0.8\textwidth, angle=90]{assets/img/Automatisiertes Temperaturwertziehen.png}
        \caption{BPMN automatisierter Ablauf Temperaturmessung}
        \label{bpmn_auto_temp}
    \end{center}
\end{figure}
\clearpage

\section{Schnittstellenbeschreibung XML-TCP RHEOGRAPH}\label{interface_description_xml-tcp}
Quelle: GÖTTFERT Werkstoff-Prüfmaschinen GmbH
\includepdf[pages=-]{assets/pdf/Kommunikation_RH25_120_Rev_R.pdf}
\clearpage

\section{Übersicht Skriptbefehle RHEOGRAPH}\label{script_order_list}
Quelle: GÖTTFERT Werkstoff-Prüfmaschinen GmbH
\includepdf[pages=-]{assets/pdf/Skriptbefehle_16.pdf}
\clearpage

\section{Projektmappenexplorer GFT40}
Quelle: GÖTTFERT Werkstoff-Prüfmaschinen GmbH
\begin{figure}[ht]
    \begin{center}
        \includegraphics[width=0.8\textwidth]{assets/img/Projektmappenexplorer.PNG}
        \caption{Projektmappenexplorer GFT40}
        \label{gft40_explorer}
    \end{center}
\end{figure}
\clearpage

\section{ListenerInterfaces UML} \label{uml_gft40}
Quelle: GÖTTFERT Werkstoff-Prüfmaschinen GmbH
\includepdf[pages=-]{assets/paradigm/GFT-40.pdf}
\clearpage

\section{GUI Haupt-Fenster}
Quelle: GÖTTFERT Werkstoff-Prüfmaschinen GmbH
\begin{figure}[ht]
    \begin{center}
        \includegraphics[width=1.0\textwidth]{assets/img/PowerToolMainGUI.PNG}
        \caption{GUI Haupt-Fenster}
        \label{gui_mainWindow}
    \end{center}
\end{figure}
\clearpage

\section{GUI Setup-Fenster}
Quelle: GÖTTFERT Werkstoff-Prüfmaschinen GmbH
\begin{figure}[ht]
    \begin{center}
        \includegraphics[width=1.0\textwidth]{assets/img/TemperatureSetupGUI.PNG}
        \caption{GUI Setup Fenster}
        \label{gui_temperatureSetup}
    \end{center}
\end{figure}
\clearpage

\section{GUI Messungs-Fenster}
Quelle: GÖTTFERT Werkstoff-Prüfmaschinen GmbH
\begin{figure}[ht]
    \begin{center}
        \includegraphics[width=1.0\textwidth]{assets/img/TemperatureMeasureGUI.PNG}
        \caption{GUI Messungs-Fenster}
        \label{gui_temperatureMeasure}
    \end{center}
\end{figure}
\clearpage

\section{Entity Relationship Modell} \label{erm_gft40}
Quelle: GÖTTFERT Werkstoff-Prüfmaschinen GmbH
\includepdf[pages=-]{assets/paradigm/GFT-40-erm.pdf}
\clearpage

\section{QS\_PR\_120} \label{qs_pr_120}
Quelle: GÖTTFERT Werkstoff-Prüfmaschinen GmbH
\includepdf[pages=-]{assets/pdf/QS_PR120.pdf}
\clearpage

\section{Umfrage Abnahme Mitarbeiter}
\begin{table}[ht]
    \begin{center}
        \begin{tabular}{| l | c | c | c |}
            \hline
            \textbf{Frage}                                                       & \textbf{+ +} & \textbf{0} & \textbf{- -} \\
            \hline
            Funktioniert die Software-Lösung im Gesamten einwandfrei?            & x            &            &              \\
            \hline
            Ist die Lösung einfach und mit einmaliger Einweisung anwendbar?      &             & x           &              \\
            \hline
            Integriert sich die Lösung gut in die bestehende Benutzeroberfläche? & x            &            &              \\
            \hline
            Stellt die Anwendung eine Arbeitserleichterung dar?                  & x            &            &              \\
            \hline
            Sehen Sie Potenzial in einer Ausweitung auf andere Geräteserien?     & x            &            &              \\
            \hline
            Halten Sie die Lösung für Ausfallsicher und wartungsarm?             & x             &           &              \\
            \hline
        \end{tabular}
    \end{center}
    \caption{Umfrage Test/Abnahme durch Mitarbeiter - EW als Blackboxtester }
    \label{umfrage_firma_ag}
\end{table}

\begin{table}[ht]
    \begin{center}
        \begin{tabular}{| l | c | c | c |}
            \hline
            \textbf{Frage}                                                       & \textbf{+ +} & \textbf{0} & \textbf{- -} \\
            \hline
            Funktioniert die Software-Lösung im Gesamten einwandfrei?            & x            &            &              \\
            \hline
            Ist die Lösung einfach und mit einmaliger Einweisung anwendbar?      &             & x           &              \\
            \hline
            Integriert sich die Lösung gut in die bestehende Benutzeroberfläche? & x            &            &              \\
            \hline
            Stellt die Anwendung eine Arbeitserleichterung dar?                  & x            &            &              \\
            \hline
            Sehen Sie Potenzial in einer Ausweitung auf andere Geräteserien?     & x            &            &              \\
            \hline
            Halten Sie die Lösung für Ausfallsicher und wartungsarm?             &              & x          &              \\
            \hline
        \end{tabular}
    \end{center}
    \caption{Umfrage Test/Abnahme durch Mitarbeiter - Produktmanager als Blackboxtester }
    \label{umfrage_firma_mk}
\end{table}

\begin{table}[ht]
    \begin{center}
        \begin{tabular}{| l | c | c | c |}
            \hline
            \textbf{Frage}                                                       & \textbf{+ +} & \textbf{0} & \textbf{- -} \\
            \hline
            Funktioniert die Software-Lösung im Gesamten einwandfrei?            &             &  x          &              \\
            \hline
            Ist die Lösung einfach und mit einmaliger Einweisung anwendbar?      & x            &            &              \\
            \hline
            Integriert sich die Lösung gut in die bestehende Benutzeroberfläche? & x            &            &              \\
            \hline
            Stellt die Anwendung eine Arbeitserleichterung dar?                  & x            &            &              \\
            \hline
            Sehen Sie Potenzial in einer Ausweitung auf andere Geräteserien?     & x            &            &              \\
            \hline
            Halten Sie die Lösung für Ausfallsicher und wartungsarm?             & x             &           &              \\
            \hline
        \end{tabular}
    \end{center}
    \caption{Umfrage Test/Abnahme durch Mitarbeiter - Software-Entwickler als Whiteboxtester }
    \label{umfrage_firma_ak}
\end{table}



