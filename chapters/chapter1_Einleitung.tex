\chapter{Einleitung}

\section{Thema}
Diese Arbeit befasst sich mit einem Vergleich der Implementierungsformen von Compilern \ac{AOT} und \ac{JIT} anhand der historischen Entwicklung, den zugehörigen Programmiersprachen, sowie den Vor- und Nachteilen. \\
Über die historische Entwicklung wird Bezug zu Entstehungsgründen und technologischer Umsetzung, sowie Verbesserung genommen. Im Kontext der Implementierungsform \ac{JIT} soll auch die Technologie des Profiling untersucht und erklärt werden, das zur Gewinnung von Performance genutzt wird.

\section{Zielsetzung}


Ziel dieser Arbeit ist die Ausarbeitung von Merkmalen und Unterschieden der beiden Implementierungsformen \ac{AOT}-Compiler und \ac{JIT}-Compiler, sowie deren Vorzügen in verschiedenen Anwendungsfällen und verschiedenen Sprachen. \\
Es soll dabei anhand der historischen Entwicklung auf Beweggründe für die Entstehung der einzelnen Techniken eingegangen werden, sowie beide unter verschiedenen Gesichtspunkten und Anwendungsfällen hinsichtlich Performance und Flexibilität verglichen werden. 
Anhand eines recherchierten Ergebnisses, soll eine Empfehlung für je nach Einsatzzweck für eine der genannten Implementierungsformen ausgesprochen werden.

% \section{Begründung}


\section{Abgrenzung}
Diese Arbeit soll kein direktes eigenes Experiment darstellen, dass einen klaren Sieger in einer fest definierten Disziplin hervorbringt, sondern ein breites Ergebnis anhand vielfältiger anderer Arbeiten präsentieren. 
Diese Ergebnisse anderer Arbeiten gilt es zu interpretieren und auf Basis dieser Schlussfolgerungen Empfehlungen auszusprechen.