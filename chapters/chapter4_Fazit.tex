\chapter{Fazit}
Abschließend kann gesagt werden, dass sowohl die Implementierungsform des \ac{AOT}-Compilers, als auch die, des \ac{JIT}-Compilers vielfältige Anwendungsgebiete haben und in der modernen Software-Architektur wichtige Rollen spielen. 
Je nach Anforderungen muss vom Entwickler eine Wahl getroffen werden, um eine Implementierungsform zu wählen, die sowohl Performance in Anbetracht vorhandener Ressourcen, als auch die nötige Flexibilität für die zukünftige Entwicklung mit sich bringt. \\
Über Kriterien, wie zum Beispiel 
\begin{itemize}
    \item Architektur/Plattform
    \item Sprachen
    \item verfügbare Ressourcen
    \item Flexibilität
    \item Update-Häufigkeit
\end{itemize}
Können Entscheidungen abgewogen werden, wobei vor allem die Sprache und Zielplattformen bestimmend sein können, wenn diese nur eine bestimmte Implementierungsform zu lassen. \\
Die Entscheidung zu einer der Implementierungsformen kann mitunter schwierig sein und ist im Normalfall irreversibel. Betrachtet man sich das Beispiel Android in \autoref{android_bsp}, so ist zu erkennen, dass die Entscheidung hin zu einer Implementierungsform auch während der weiteren Entwicklung immer wieder überdacht wird und hier tatsächlich versucht wird, die Vorzüge aus beiden Implementierungsformen zu vereinen. 